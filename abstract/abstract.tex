\newpage

\center\boxed{\textbf{Abstract}}

Dans ce TIPE, on s'intéresse aux liens entre le raisonnement en
mathématiques et la programmation informatique.

En effet lorsque l'on fait un raisonnement en mathématiques, les objets
introduit ne sont, dans beaucoup de cas, pas définis \emph{de manière
strictement claire}, tandis qu'en informatique on utilise des opérateurs
de définitions.

On pourra alors se demander comment un ordinateur pourrait-il faire des
transformations telles que des simplifications d'expressions mais sans
avoir de valeur claire pour les variables introduites, et ainsi simuler
des preuves mathématiques.

\newpage

\section*{References}\label{references}
\addcontentsline{toc}{section}{References}

\phantomsection\label{refs}
\begin{CSLReferences}{1}{0}
\bibitem[\citeproctext]{ref-coqencoq}
BARRAS, Bruno. 1996. {``Coq En Coq.''}
\url{https://inria.hal.science/inria-00073667/document}.

\bibitem[\citeproctext]{ref-cwinl}
KLOP, Jan Willem. 2000. \url{https://ir.cwi.nl/pub/6129/6129D.pdf}.

\bibitem[\citeproctext]{ref-lmaranget}
MARANGET, Luc. 2008.
\url{https://www.cs.tufts.edu/~nr/cs257/archive/luc-maranget/jun08.pdf}.

\bibitem[\citeproctext]{ref-sometapas}
MESEGUER, José. n.d.
\url{https://maude.cs.illinois.edu/w/images/7/70/Maude-tapas.pdf}.

\end{CSLReferences}
